\documentclass{acm_proc_article-sp}
\usepackage{graphicx}
\usepackage{pifont}% http://ctan.org/pkg/pifont
\newcommand{\cmark}{\ding{51}}%
\newcommand{\xmark}{\ding{55}}%
\newcommand\floor[1]{\lfloor#1\rfloor}


\begin{document}

\title{Quantifying The Affects of Context Switching On Programmer Productivity}

%
% You need the command \numberofauthors to handle the 'placement
% and alignment' of the authors beneath the title.
%
% For aesthetic reasons, we recommend 'three authors at a time'
% i.e. three 'name/affiliation blocks' be placed beneath the title.
%
% NOTE: You are NOT restricted in how many 'rows' of
% "name/affiliations" may appear. We just ask that you restrict
% the number of 'columns' to three.
%
% Because of the available 'opening page real-estate'
% we ask you to refrain from putting more than six authors
% (two rows with three columns) beneath the article title.
% More than six makes the first-page appear very cluttered indeed.
%
% Use the \alignauthor commands to handle the names
% and affiliations for an 'aesthetic maximum' of six authors.
% Add names, affiliations, addresses for
% the seventh etc. author(s) as the argument for the
% \additionalauthors command.
% These 'additional authors' will be output/set for you
% without further effort on your part as the last section in
% the body of your article BEFORE References or any Appendices.

\numberofauthors{5} %  in this sample file, there are a *total*
% of EIGHT authors. SIX appear on the 'first-page' (for formatting
% reasons) and the remaining two appear in the \additionalauthors section.
%
\author{
% You can go ahead and credit any number of authors here,
% e.g. one 'row of three' or two rows (consisting of one row of three
% and a second row of one, two or three).
%
% The command \alignauthor (no curly braces needed) should
% precede each author name, affiliation/snail-mail address and
% e-mail address. Additionally, tag each line of
% affiliation/address with \affaddr, and tag the
% e-mail address with \email.
%
% 1st. author
\alignauthor
A\\
       \affaddr{Blah Blah}\\
       \affaddr{Blah Blah}\\
       \affaddr{Blah, Blah}\\
       \email{blah@blah.com}
% 2nd. author
\alignauthor
B\\
       \affaddr{Blah Blah}\\
       \affaddr{Blah Blah}\\
       \affaddr{Blah, Blah}\\
       \email{blah@blah.com}
% 3rd. author
\alignauthor
C\\
       \affaddr{Blah Blah}\\
       \affaddr{Blah Blah}\\
       \affaddr{Blah, Blah}\\
       \email{blah@blah.com}
\and  % use '\and' if you need 'another row' of author names
% 4th. author
\alignauthor
D\\
       \affaddr{Blah Blah}\\
       \affaddr{Blah Blah}\\
       \affaddr{Blah, Blah}\\
       \email{blah@blah.com}
% 5th. author
\alignauthor
E\\
       \affaddr{Blah Blah}\\
       \affaddr{Blah Blah}\\
       \affaddr{Blah, Blah}\\
       \email{blah@blah.com}
}


\maketitle
\begin{abstract}
Abstract goes here ...
\begin{itemize}
\item{Generic statement: context switching good/bad}
\item{The problem: how do we know what types of switching is googd or bad for programmer productivity}
\item{The solution: correlation analysis}
\item{Goal statement}
\item{Methodology}
\item{Results}
\item{Final take away}
\end{itemize}
\end{abstract}

% A category with the (minimum) three required fields
\category{H.4}{Information Systems Applications}{Miscellaneous}
%A category including the fourth, optional field follows...
\category{D.2.8}{Software Engineering}{Metrics}[complexity measures, performance measures]

\terms{Theory}

\keywords{programmer productivity, context switching, interleaving, software projects}

\input{intro}

\section{Background and Related Work}
In this section we define the necessary concepts used in our paper and prior academic studies that are related to this paper.

%\subsection{Brief Overview of Relevant Software Artifacts}
%Codealike tracks programmer related information for two IDEs namely, Eclipse, and Microsoft Visual Studio (MSVS). In our study we conduct empirical analysis on a dataset that corresponds to the programmers' activities tracked in MSVS.  In MSVS, software artifacts that are necessary to create an application belong to a container called \textit{solution}. In MSVS, an application can be created in a certain programming languages such as C\#, Visual Basic, and J\#. A solution can consist multiple \textit{projects}. A project corresponds to a virtual directory hierarchy and all the files belonging to the project. These files can be configuration files, source code files written in a certain programming language, test cases etc. A project also contains the necessary file system paths to the necessary items and build settings. For creating an application if an object-oriented programming language is chosen such as C\#, then source files can belong to different \textit{namespaces}. A namespace is an object-oriented organization concept that allows for organization of different program elements namely, variables, methods, classes etc. Conceptually namespaces in C\# or C++ are same as \textit{packages} in Java or ADA ~\cite{kurtev:namespace}.

\subsection{Codealike}
Codealike \footnote{http://codealike.com/} is an automated tool for personal software process. Codealike can be added the Microsoft Viual Studio (MSVS) IDE as a plugin. Codealike can track different software artifacts relevant to MSVS such as projects, namespaces, and files as well as programming activities such as coding, debugging, and building. In our study we used a dataset that consisted programmingtraces of different programmers all across the world that use Codealike. 

The dataset that we use in our study is a JSON file that included programming related information of different programmers, as JSON objects. Each of these JSON objects represent a programmer's activity traces. Each JSON object contains a set of fields that represent a programmer's activity trace. 

The `sessions' field is a collection of `session' objects. A `session' object is a collection of properties such as, `SessionID', `Path', and `State'. The `Path' property represents the relative path inside the MSVS IDE, whereas, `State' corresponds to one of the seven programming states that a programmer have used. The `Path' field is a string of the format ``a:b:c:d:e'' where `a' represents a project context or a domain. A `project context' corresponds to a virtual directory hierarchy that contain all the files related to the project such as configuration files, source files, and test cases. `b' represents the name of the associated file, `c' represents the name of the associated `namespace', `d' represents the associated `class', and `e' represents the name of the associated member.  The `State' field is a numeric value where each a certain value indicates a specific state. The states None, Idle, System, Code, Debug, and Build are respectively represented by numerals -1, 0, 1, 2, 3, 4, and 5.
%Except for `Path', all the fields are numeric. `E\_StartTicks', and `E\_EndTicks' is measured in micro-ticks that is equal to $10^{-7}$ micro-ticks. 
%Codealike tracks six programming related events inside a MSVS IDE. They are: tracking focus of the mouse for a file, tracing if a file is being changed or not, tracing a solution being opened, tracing a solution being closed,  tracing the status of building a solution, tracing the status of building a project.

%\subsection{Dataset}


\subsection{Assumption and Definitions}
The following concepts are related to our study and we define them as following:
\subsubsection{Programming Related States}
A programming related state refers to a specific state of a programmer that is related to software development.  Codealike considers seven states namely, `None', `Idle', `System', `Code', `Debug', and `Build'. In addition to these six states, we also consider an additional programming state called `Navigation'. Codealike does not explicitly track this state and we use our own methodology to detect navigation activities from the dataset of interest.    
%In our paper we consider the following states that we consider in this paper
%are `idle`, `system`, `code`, `debug`, and `build`.
We refer to a programming related state as `state' for the rest of this paper.
\subsubsection{Session}
A session is a collection of a programmer's programming related activities for a \textit{specific} period of time.
\subsubsection{Time Window of Interest}
Time window of interest refers to a specific snapshot of a complete session and is measured in minutes. We refer to time window of interest as window for the rest of this paper. Sessions can vary widely in duration depending on how programmer activity traces were collected. To account such wide range variability and simplify our analysis we use window as the measurement unit in our study. Sessions are divided into segments that have the same duration as the window duration. We use windows answer all the three research questions. In our analysis we use window as parameter and vary this parameter in terms of duration such as one minute, five minutes and ten minutes. As shown in Section 3, each session is divided into segments based on the duration of each selected window.
\subsubsection{Programmer Productivity}
In this paper programmer productivity measures how frequently a programmer maintains a specific set of programming related states for a certain time window of interest. We refer to programmer productivity as productivity for the rest of this paper.

\subsubsection{Context Switching}
Context switching is the actiity of switcing from one artifact or one programmign state to another. In short we refer to context switching as `switching', and each type of context switches such as domain switches and file switches as `switches' in short. Switches can be divided into two broad categories: artifcat and state. Artifact switches correspond to switches related to programmer's relative path inside a MSVS IDE. Satte switches relate to programmer's switching between one programming state to another.   

In our study we assume that between two segemnts $s1$ and $s2$ only one type of artifact related switching such as domain switches and file switches can happen. For example, between $t1$ and $t2$ either a domain switch or a file switch can happen but not both. 

\subsection{Related Work}

\begin{itemize}
\item{ ~\cite{lungu:devinfoneed} }
\item{ ~\cite{kevic:banita:changetask} }
\item{ ~\cite{robilliard:task:icpc} }
\item{ ~\cite{Parnin:Resumption} }
\item{ ~\cite{Ko:tse} }
\end{itemize}       






\section{Methodology}
We organize this section as follows: first we provide details 
on how we extract programmer activity traces for each programmer from the dataset.
Next we present hoe we calculate programmer productivity as calculation of programmer productivity is relevant to answer all the three research questions. 
Then we provide details on actions that we took to answer the three research questions.   
%Figure~\ref{meth} presents the steps that we use to perform this empirical study. 
%
%
%\begin{figure}[htbp]
%\centering
%\includegraphics[scale=0.30]{img/meth.PNG}
%%\includegraphics[scale=0.30]{img_pdf/meth.pdf}
%\caption{Main Steps of the Research Study}
%\label{meth}
%\end{figure}

\subsection{Data Extraction}
\label{dataextract}
%In our paper we use a dataset that contain programmers' interactions in a MSVS
%environment as they were performing software development tasks. 
We observe that analyzing the dataset of interest ina raw format is non-trivial, as all the activity traces of all the programmers are compressed into one file. We used a parsing program written in JavaScript, that parses the dataset and spits the entire dataset into activity trace files for all the programmers to extract the activity traces of each programmer. The JavaScript program uses the `name' field to identify the name of the programmer, and the `sessions' fields to the programmer's activity traces. Furthermore, we used the `Path' property within each `session' field to determine the types of files used by programmers. Empirical findings related to this step are presented in the next section.  

\subsection{Calculating Programmer Productivity}
\label{productivity}
We use window as a measurement unit to calculate programmer productivty 
for multiple sessions.  
We rely on prior work to quantify programmer productivity. Similar to Corley et al. [reff-Corley], we use a growth function to calculate 
programmer productivity. The growth function is presented as following: 
\begin{equation}
f = C / (1 + A * exp^{ -2 * B * count} )
\end{equation}
\begin{equation}
offset = C / (1+A)
\end{equation}
\begin{equation}
productivity = (( f - offset ) / (C - offset ) ) * 100
\end{equation}
Here, C = 100, A = 8, B = 0.02, and count = count of states that happened in
a particular segment. Please recall that a segment is part of a session. 
To calculate programmer productivity score for a session we divide the session into $\floor{session length/window length}$ segments. The growth function takes the count of programming states that occur in a certain segment, and gives a productivity score for the segment. We derive the productivity score of a session by taking the average of the productivity scores for all segments within a session. 

If a session has a productivity score of at least one then we term that session as a `highly productive' session, other wise it is a `low productive' session. As all our research questions are related to 
more productivity of programmers we use `high productive' sessions throughout the paper. 

%\subsection{Data Preprocessing}
%\label{dataprep}
%The collected dataset is 1.2 Gigabyte JSON file that contains all the related %programming traces of 339 programmers, collected by Codealike. To get the relevant %programming traces for each of these 339 programmers we use a Javascript program that %parses the entire JSON file and splits out 339 individual JSON files. Each of these 339 %JSON files correspond to the programming traces of 339 different programmers, and are %stored in a directory. In the next phases of our analysis we used this set of 339 JSON %files. As each of these 339 files correspond to the programming activity traces of 339 %different users, for the rest of the paper we refer to each of these 339 JSON files as %\textit{users}. The collected dataset has 339 users. 

%We organize the rest of the section with respect to the research questions of interest. %In each of the following subsections we describe the steps to answer the three research 
%questions of interest. We use Python scripts to implement all the other steps of our %study. These scripts are available online\footnote{https://github.ncsu.edu/CSC510-%Fall2014/Empirical-CodeALike}.

%\subsection{RQ1}
%Our study focuses on the effects of different types of context switching on programmer productivity. The first step of performing this
%investigation is to categorize all sessions into different states. The
%motivation of performing this particular activity is to gain insight on the
%distribution of states across all sessions. This insight can tell us the states
%that the programmers' use, and provide a better understanding of the state
%switching that occurs for a certain session.
%To categorize all the sessions for a user we perform the following steps:
%\begin{itemize}
%\item{for each session:
%\begin{itemize}
%\item{for each state of interest namely, none, idle, system, code, debug, build, and navigation
%\begin{itemize}
%\item{we calculate the total count of the state occurring for this session }
%\item{we calculate the total duration corresponding to the state occurring for
%this session using `E\_StartTicks', and `E\_EndTicks' }
%\item{we calculate \textit{time\_per\_state} for each state of interest, where
%\textit{time\_per\_state} = \textit{total duration of state}/\textit{total count of state}}
%\end{itemize}
%}
%\item{after calculating \textit{time\_per\_state} for all states of interest we
%designate a state as a \textit{leader state}, for this session if it has the
%largest \textit{time\_per\_state} value.}
%\item{The session is categorized by its \textit{leader state}.}
%\end{itemize}
%}
%\end{itemize}
%We perform the above mentioned 
%steps to categorize all sessions that exist across all 339 users.
%
%To illustrate our steps we use an imaginary example. Lets's assume for an
%imaginary session, the \textit{time\_per\_state} values for the states None,
%Idle, System, Code, Debug, Build, and Navigation respectively is 
%1.1, 2.5, 0.4, 12.18, 3.6,
%25.7, and 7.23. According to our methodology the \textit{leader state} is Build, and this
%imaginary session will be categorized as a `Build' session. 
%
%Please note that the `Navigation' state is not explicitly tracked by Codealike, and we use our
%own methodology to trace `Navigation' activities of 339 users. To trace these navigation activities of a user our methodology detects if the `Path' field is changed. \textit{Total count of navigation} refers to the total count of the `Path' field being changed for one session. \textit{Total duration of navigation}
%refers to the total duration of the `Path' field being changed using the `E\_StartTicks', and `E\_EndTicks' fields.
%
%\subsection{RQ-2: How can productivity of programming sessions be measured?}
%The second step to measure the effects of different types of context switching on programmer productivity is to assemble a set of actions to measure programmer productivity. To measure programmer productivity we count of the programming states that occurred for a specific window across different sessions for a programmer. Next we use the following equations to calculate productivity during that window. The equation has one variable which is the count of programming states.
%[need info on how we got this equation - Dr. Parnin]
%\begin{equation}
%f = C / (1 + A * exp^{ -2 * B * count} )
%\end{equation}
%\begin{equation}
%offset = C / (1+A)
%\end{equation}
%\begin{equation}
%productivity = (( f - offset ) / (C - offset ) ) * 100
%\label{prod}
%\end{equation}
%Here, C = 100, A = 8, B = 0.02, and count = count of states that happened in
%that particular window. We label window as $n$ which is measured in minutes and
%can be of different lengths such as five, 10, 15, and 30 minutes.
%To summarize we perform the following steps to calculate productivity for each
%session:
%\begin{itemize}
%\item{for each session:
%\begin{itemize}
%\item{we divide the session into buckets of $n$ minutes; e.g. if we have a
%session of length 10 minutes, then for $n=5$ we will have two buckets for that
%session}
%\item{for each snapshot we calculate the productivity using
%equation~\ref{prod}}
%\item{to calculate the productivity of a session we take the average
%productivity values of all the buckets}
%\item{we assign the average productivity of all buckets for a session, as the
%productivity for that particular session}
%\item{we label a session's productivity as `high' if the productivity score is greater than or equal to one}
%\item{we label a session's productivity as `low' if the productivity score is less than one}
%\end{itemize}
%}
%\end{itemize}
%If a session's duration is less than $n$, then we assign zero for that session's
%productivity score. 
%
%To illustrate our methodology of calculating productivity we
%use an imaginary example. Let us assume we want to determine the productivity of
%a session that has a duration of 10 minutes. Let us assume we assuming window,
%$n=5$ here. We will have two buckets for this imaginary session, and we will
%take the average of the productivity values that we obtain for these two
%buckets. For example, if the productivity score for these two scores are 11, 31
%then productivity score for the session is 21. Please note that if we select $n
%> 10$ for the same imaginary session, our methodology will assign zero
%productivity score. In our study we consider five windows of different duration: a window of one minute, a window of five minutes, a window of 10 minutes, a window of 15 minutes, and a window of 30 minutes.

\subsection{RQ1}
\label{switches}

Using our background knowledge, and assumptions presented in Section 2, we hypothesize that two fields `Path', and `State' can be used to detect different types of context switching based on the following observations: 
\begin{itemize}
\item{As `Path' presents a programmer's use of a certain artifact at a certain point of time, we can use this property to trace how may time the programmer has moved between different types of artifacts such as domain, file, and namespace.}
\item{As `State' presents a programmer's use of a certain programming action, we can 
use this property to trace how many times the programmer has moved between different programming actions such as code, debug, and build.}
\end{itemize}  
The `Path' property contains information related to five different artifacts implying the possibility of detecting five different types of context switches from the `Path' property. We exclude a certain type of context switch if it is too fine grained to infer useful information out of it. We observe that the two artifacts `class', and `member' are too fine grained to useful information relevant to context switching, and thus we exclude these two potential types of context switches from our study. 

We use windows as our measurement unit for each session to calculate the amount of various types of switches. We maintain seperate counters for each potenital switch. We first divide each session into $\floor{session length / window length}$ segments to quantify the different context switches. For each segment we extract the values for class, domain, file, member, namespace, and state properties. 
If the values are different for the `State' property then we detect a state switch, and increment the counter accordingly. Considering our assumption of 
artifact related switches, we first compare the values of domains first. If there is a difference then we increment the corresponding counter, and do not consider any other related switches. We consider detecting and quantifying file switches if the values for doamin are same across two segments. In the same manner we detect and quantify namespace, class, and member switches if the values for file, namespace, and class, respectively are the same. We detect artifact switches in this manner to be consistent with the organization of artifacts inside a MSVS IDE.   
 
The corresponding counters for each switches remain unchanged if there is no difference or the value for a specific switch is an empty. We divide the count
for each type of switch by the number of segments to determine the amount of 
switching for each context type. We repeat the above mentioned procedure for five windows of different durations: one minut, five minutes, 10 minutes, 15 minutes, 
and 30 minutes.     

We use an imaginary example to illustrate even further. 
Let us assume we have a session of length 15 minutes. 
If we use a window of five minute duration we then we will have three segments. 
If the values of the `Path' for the first, second, and third segment 
respectively is "a:b:c:d:e", "v:b:x:d:z", and "v:c:p:q:r", where each of the literals are non-empty then  the amount of domain, file, and namespace will be 
0.33, 0.33, and 0.00, respectively. For same session, if state values for the three segments are respectively 2, 3, and 4, then we the amount of state switching is 0.67.   

  


\subsection{RQ2}
\label{effect}
Answering RQ2 requires quantifying the occurrences of different types of context switches, calculating programmer productivity, and applying a technique that will illustrate the relationship between productivity and different types of context switches. The relationship will highlight the effect of different types of context switches on programmer productivity. Previously in this section we have presented how we calculate the occurrences of various types of context switches in Section ~\ref{switches}, and programmer productivity for a session in Section ~\ref{productivity}. We use the same procedure here to calculate occurrences of different types of context switches, and programmer productivity. 

Academicians have previously used correlation measure such as Pearson, Spearman, and MIC to investigate the effect of one variable on another [reff needed]. Pearson and Spearman correlations capture the linear relationship between two statistically dependent variable, whereas, MIC captures the non-linear relationship. Taking inspiration from prior work that have used correlation measures we use three correlation metrics namely, MIC, Pearson, and Spearman to investigate any existing effect of different types of context switches and programmer productivity. Along with correlation co-efficients 
we also record the p-values because with respect to 
correlation coefficients p-values state whether or not a correlation is happening by chance, and not for the relationship between two variables[reff. needed].      

As effects of different types of context switching can be different for different programmers we observe the necessity of conducting analysis in two stages considering the fact. In the first stage we focus on an individual programmer and investigate what are the effects of various context switches 
on different levels fo programmer productivity. In the second stage we investigate if 
the affects are same or different for all the programmers in the dataset. In the following two subsections we discuss how we attempt the two stages of investigation. 
In both stages we apply our knowledge of calculating programmer productivity, and occurrences of different types of context switching.
 
\subsubsection{Effects of Context Switches for Individual Programmers}
We performed the following steps to quantify the effects of context switching for 
individual programmers as following:
\begin{itemize}
\item{First, we create a list that holds the productivity values for all sessions
which had a productivity score greater than or equal to one. For convenience we refer to this list as \textit{high productivity vector}. }
\item{Then, for each type of context switching we create two separate lists: one list corresponds to the occurrences of a specific context switch for sessions that have a productivity score of at least one. For convenience we refer to this list as \textit{high switch\_x vector}, where x refers to a certain context switch such as `class', `domain', and `file'. Thus if there are three types of context switches then we will create six lists for each programmer.}
 
\item{We apply three correlation measures Pearson, Spearman, and MIC between high productivity vector and one of the\textit{high switch\_x vector}. Apply the same procedure for all type of switches.}

\end{itemize} 

\subsubsection{Effects of Context Switches for All Programmers}
From the previous subsection we obtained a set of lists 
for each programmer. We use these set of lists to calculate the effects of context switches for all programmers. The steps we follow are provided below:
\begin{itemize}
\item{First, we combine the \textit{high productivity vectors} of all programmers into one list and call it as \textit{overall high productivity vector}. }
\item{Then, for each type of context switching we create two separate lists: we create the \textit{overall high switch\_x vector} by merging the \textit{high switch\_x vectors} of all programmers. }
 
\item{We apply three correlation measures Pearson, Spearman, and MIC between \textit{overall high productivity vector} and one of the\textit{overall high switch\_x vector}. We apply  the same procedure for all type of switches i.e. if there are three types of switches then we will apply the same procedure for all of the three types of context switches.}

\end{itemize}

\subsection{RQ3}

In Section ~\ref{effect} we have considered the effects of various switches to be linear as well as non-linear and used different metrics to calculate those effects. For consistency we answer RQ3 in the same manner. Our underlying methodology to investigate the switches that are advantageous or disadvantageous remains the same for both liner and non-linear perspectives; the difference lies within the use of correlation co-efficient i.e. with respect to linearity we use Pearson and Spearman as correlation coefficients, and MIC for non-linearity. We present the procedure as following:  

For a certain window, a context switch is advantageous if both conditions \textit{C1-A}, and \textit{C2-A} are true. Similarly, a context switch is advantageous if both conditions \textit{C1-D}, and \textit{C2-D} are true. The conditions \textit{C1-A, C2-A, C1-D, C2-D} are presented as following:   

\begin{itemize}
\item{\textit{C1-A}: the correlation co-efficient between the \textit{overall high productivity vector} and the \textit{high switch\_x vector} is greater than or equal to 0.20. In case of linearity measure 
we will use two correlation coefficients Pearson, and Spearman, and for both these correlation coefficients the condition must be true.}

\item{\textit{C2-A}:the p-value for the obtained correlation coefficient is less than or equal to 0.05}

\item{the correlation coefficient between the \textit{overall high productivity vector} and the \textit{high switch\_x vector} is less than or equal to -0.20. In case of linearity measure we will use two correlation coefficients Pearson, and Spearman, and for both these correlation coefficients the aforementioned condition must be true.}

\item{the p-value for the obtained correlation coefficient is less than or equal to 0.05}
\end{itemize}

If a context switch is not marked as advantegous or disadvantageous then we mark it as an `inert' switch.  We apply the above mentioned procedure five windows of durations: one minute, five minutes, 10 minutes, 15 minutes, and 30 minutes for all context switches of interest. 

If a switch is marked as advantageous for three out of the windows of interest then we mark that switch as advantageous for productivty when programmers are more productivty. 
Similarly a switch is marked disadvantageous or inert if the switch is marked as disadvanatageous or inert respectively for three out of the five windows of interest. 
 

 


\section{Results}

We organize this section as following: first we present summary findings that we extracted from the dataset to provide an overview of the dataset in use. Next, we answer the three research questions in three different subsections. 
The total size of the dataset is 1.20 Gigabytes. We extracted the programming traces of 339 programmers using our JavaScript program as mentioned in Section 3. After extracting the dataset we observe that there are 84, 030 programming sessions. 
According to our analysis 95\%, of the files used by the 339 programmers were C\# files.   
We also observe that the count of high productive sessions decrease with increase in window length. For example, according to Figure~\ref{high_prod_sess} the count of `high' productive sessions are lowest when a window duration is 30 minutes. We also observe that the count of `high' productive sessions starts to decrease when we increase the window duration for five minutes or more. 

\begin{figure}[htbp]
\centering
\includegraphics[scale=0.30]{img/general_prod.png}
\caption{Count of `high' programming sessions vary with window size}
\label{high_prod_sess}
\end{figure}

In the following subsections we provide empirical findings related to the research questions of interest.  
\subsection{RQ1}
Applying our methodology identify four types of switching that are presented below with definitions.
\begin{itemize}
\item{Domain Switching: We define the activity of switching from one domain to another as domain switching.}
\item{File Switching: We define the activity of switching from one file to another for inside a domain as file switching.}
\item{Namespace Switching: We define the activity of switching from one namespace to another within a file as namespace switching.}
\item{State Switching: We define the activity of switching from one state to another as state switching.}
\end{itemize}

We use Figure~\ref{rq1-mean} to present the mean occurrence amount of each of the four identified context switches. The X and Y axis of Figure ~\ref{rq1-mean} presents the type of context switch, and the mean amount of context switch respectively. We illustrate the variability in the amount of occurrences for each type of switch in Figure~\ref{rq1-var-5}, and ~\ref{rq1-var-15}. Due to space constraints we only present relevant findings for two windows of duration five and 15 minutes, respectively. 
 
\begin{figure}[htbp]
\centering
\includegraphics[scale=0.30]{img/rq1-mean.png}
\caption{Amount of different types of context switches}
\label{rq1-mean}
\end{figure}

\begin{figure}[htbp]
\centering
\includegraphics[scale=0.30]{img/rq1-5.png}
\caption{Amount of different types of context switches}
\label{rq1-var-5}
\end{figure}

\begin{figure}[htbp]
\centering
\includegraphics[scale=0.30]{img/rq1-15.png}
\caption{Amount of different types of context switches}
\label{rq1-var-15}
\end{figure} 
 
%Next in Table~\ref{RQ1} we answer the seocnd part of RQ1 by presenting relvant empirical findings. We sumamrize these findigns using a tuple 
%in each cell that presents the mean, median, and standard deviation of the frequency 
%for a certain type fo context switch. Fro example, from Table~\ref{RQ1} 
%we observe that the mean, median, and standard deviation of file switching 
%occurarneces respectively is x, y, and z for a five minute window in session where programmers are more porductive.  
%
%\begin{table}
%\caption{Amount of different types of context switches}
%\label{RQ1}
%%\small
%\tabcolsep=0.11cm
%\begin{tabular}{|c|c|c|c|c|}
%  \hline
%   & Domain  & File  & Namespace  &  State  \\
%  \hline
%  1 & (3.97,2.00,7.15) & (0.73,0.60,0.77) & (0.16,0.0,0.34) & (0.74,0.57,0.74)\\
%  \hline
%  5 & (13.03,8.00,16.87) & (4.36,4.00, 2.79) & (1.38,1.00,1.38) & (3.31,3.05,1.99)\\
%  \hline
%  10 & (23.73,15.5,26.95) & (9.09,8.50,4.82) & (3.16,2.79,2.37) & (5.79,5.68,3.08)\\
%  \hline
%  15 & (32.38,20.78,36.53) & (13.80,13.00,6.75) & (4.94,4.43,3.37) & (7.59,7.56,3.94)\\
%  \hline  
%  30 & (44.19,28.44,46.21) & (26.24,25.50,13.09) & (9.43,9.00,5.77) & (11.02,11.00,6.33)\\
%  \hline    
%  \end{tabular} 
%\end{table} 

From Figures~\ref{rq1}, ~\ref{rq1-var-5}, and ~\ref{rq1-var-15} we
observe that the amount of domain switches is larger and more variable compared to the other types of switches. From Figure~\ref{rq1} we observe that the mean amount of project switches, and namespaces switches is respectively the largest, and smallest amongst all four switches, for all windows of interest. From both Figures~\ref{rq1-var-15, rq1-var-15} we observe that with respect to number of outliers and spread, the amount of project switches is more variable compared to the other three switches. 



\subsection{RQ2}

We answer RQ2 providing empirical findings 
in the following two subsections. First we present our findings for individual programmers, then we present our findings for all 339 programmers. 
\subsubsection{Effects of switches for individual programmers}
We realize that presenting the effects of the four types of switches on high productivity is non-trivial, and therefore we present a summarized version of our findings in Figures~\ref{rq2-5} and ~\ref{rq2-15}. We use Figures~\ref{rq2-5} and~\ref{rq2-15} to observe if the effects of different switches are similar for majority of the programmers. Such observation can also hint at the generalizability of the effect amongst different programmers. Figures~\ref{rq2-5} and ~\ref{rq2-15} correspond to 
the findings for a window duration of five minutes, and 15 minutes respectively. 

\begin{figure}[htbp]
\centering
\includegraphics[scale=0.30]{img/rq2-5.png}
\caption{Effects of switches: five minute window}
\label{rq2-5}
\end{figure}

Both Figures~\ref{rq2-5} and ~\ref{rq2-15} present the count of programmers belonging to three categories for each of the four switches: `significant', `moderate', or `negative'. `Significant', `Moderate', and `Negative' respectively presents the count of programmers for which the Pearson correlation coefficient is  at least 0.50, positive but less than 0.50, and less than zero. The p-value for all the three categories are less than or equal to 0.05. Please recall from Section 3 that we applied Pearson correlation coefficient between the \textit{high productivity vector} and each of the \textit{switch\_x vectors}.  

\begin{figure}[htbp]
\centering
\includegraphics[scale=0.30]{img/rq2-15.png}
\caption{Effects of switches: 15 minute window}
\label{rq2-15}
\end{figure}


From Figure~\ref{rq2-15} we observe that 84 and 70 programmers respectively exhibit `Significant', and `Moderate' trends for file switches. For file switching the count of programmers drops drastically to 8 for `negative' category. Namespace switching 
also shows similar trends, where the count of programmers is least for the `negative'
category. The trend is inverse for state switches; for this switch the count of programmers are highest for `negative' amonhgst the three categories.  

The observations from Figure~\ref{rq2-15}, is applicable for Figure~\ref{rq2-5} as well. According to Figure~\ref{rq2-5} we observe that majority of the programmers fall in the `significant', and `moderate' categories, for file and namespace switches. For state switches the trend is opposite; the count of programmers belonging to the `negative' category is bigger compared to `significant', and `moderate'. Our observations from Figures~\ref{rq2-5} and ~\ref{rq2-15} suggest that effects of file switches are similar to namespace switches, but opposite to state switches. From both figures we also observe that domain switches do not exhibit any specific trend unlike file, namespace, or state switches.       
  




\subsubsection{Effects of switches for 339 Programmers}  

In Table~\ref{general-corr} we present our findings with respect to general correlation for two different window durations. For space constraints we present our findings for windows of five minutes and 15 minutes. In Table~\ref{general-corr}, the column `Corr. Coeff' presents the correlation co-efficient for a one of the four types of context switching namely, domain switching, file switching, namespace switching, and state switching. The column `P-Value' column presents the p-value for that particular correlation co-efficient. For example, considering window duration of five minutes 
the Pearson and Spearman correlation coefficient for file switches is respectively 0.45, and 0.45.  

\begin{table*}[htp]
\caption{What are the effects of different types of context switches on high productivity?}
\label{general-corr}
\small
\begin{tabular}{|p{1.7cm}|c|c|c|c|c|c|c|c|c|}
  \hline
   & \multicolumn{4}{c|}{Five Minutes}  &  \multicolumn{4}{c|}{15 Minutes} \\
  \hline
  Type of Switch & \multicolumn{2}{c|}{Pearson}  &  \multicolumn{2}{c|}{Spearman} & \multicolumn{2}{c|}{Pearson}  &  \multicolumn{2}{c|}{Spearman} \\
  \hline
  Type & Corr. Coeff. & P-Value & Corr. Coeff. & P-Value & Corr. Coeff. & P-Value & Corr. Coeff. & P-Value \\
  \hline
  Domain  & 0.38 & 0.0 & 0.28 & 0.0 &  -0.03 & 0.03 & -0.09 & $1.29*10^{-07}$  \\
  \hline
  File  & 0.45 & 0.0 & 0.45 & 0.0 & 0.34 & $1.26*10^{-99}$ & 0.30  & $3.75*10^{-76}$  \\
  \hline
  Namespace  & 0.42 & 0.0 & 0.39 & 0.0 & 0.28 & $6.87*10^{-65}$ & 0.23 & $1.47*10^{-45}$  \\
  \hline
  State  & 0.00 & 0.83 & -0.10 & $1.74*10^{-56}$ & -0.36 &  $8.50*10^{-108}$ & -0.40 & $1.93*10^{-135}$  \\
  \hline
  \hline
  \end{tabular}
\end{table*}

From Table~\ref{general-corr} we observe empirical findings that are similar to what we observed at individual level. We observe that for both windows of different duration lengths, both file switches and namespace switches 
exhibit positive correlations. On the other hand, state switches exhibit negative correlation for the 15 minute window duration, and also for five minute duration ignoring the Pearson coefficient with a p-value >0.05. Similar to our observations studying individual programmers, we cannot conclude the effect of domain switching as it exhibits both positive and negative correlations for different window durations. Observations from Table~\ref{general-corr}, and Figures~\ref{rq2-5},~\ref{rq2-15} hint that file switches and namsepace switches might have a positive effect on high productivity, whereas the effect of state switches might be negative for high productivity.          



\subsection{RQ3}

We use table ~\ref{rq3} to present our findings related to RQ3. In Table~\ref{rq3}, `A',`D', and `I' respectively denotes `Advantageous', `Disadvantageous', and `Inert'. Table ~\ref{rq3} presents how the four of the context switches are indentified in terms of being advantegous, disadvantageous, or inert. We used a set of conditions to determine whether or not a switch is advatageous or disadvantageous. If a condition is satisfied then we place a \cmark for that particular window and context switch. If a condition is not satisfied then place a \xmark. Table ~\ref{rq3} includes the findings for linearity where we used Pearson and Spearman correlation coefficients. For each window size we determine if a context switch is advantageous or disadvantageous indicated in the \textit{Decision} row of Table~\ref{rq3}. The final decision on whether or not a context switch is advantageous or disadvantageous is presented in the \textbf{Final decision} row.  
 
\begin{table}[htp]
\caption{Identifying beneficial and detrimental context switches}
\label{rq3}
\small
\begin{tabular}{|p{1.3cm}|p{1.3cm}|p{1.3cm}|p{1.3cm}|p{1.3cm}|p{1.3cm}|}
  \hline
    & Domain switching & File switching & Namespace switching & State switching \\
  \hline
  One Minute &  
  \shortstack{C1-A: \cmark \\ C2-A: \cmark \\ C1-D: \xmark \\ C2-D: \xmark} & 
  \shortstack{C1-A: \cmark \\ C2-A: \cmark \\ C1-D: \xmark \\ C2-D: \xmark} & 
  \shortstack{C1-A: \cmark \\ C2-A: \cmark \\ C1-D: \xmark \\ C2-D: \xmark} & 
  \shortstack{C1-A: \cmark \\ C2-A: \cmark \\ C1-D: \xmark \\ C2-D: \xmark} \\
  \hline          
  \textit{Decision} & \textit{A} & \textit{A} &
  \textit{A} &  \textit{A} \\     
  \hline
  Five Minutes &  
  \shortstack{C1-A: \cmark \\ C2-A: \cmark \\ C1-D: \xmark \\ C2-D: \xmark} & 
  \shortstack{C1-A: \cmark \\ C2-A: \cmark \\ C1-D: \xmark \\ C2-D: \xmark} & 
  \shortstack{C1-A: \cmark \\ C2-A: \cmark \\ C1-D: \xmark \\ C2-D: \xmark} & 
  \shortstack{C1-A: \xmark \\ C2-A: \xmark \\ C1-D: \xmark \\ C2-D: \xmark} \\
  \hline          
  \textit{Decision} & \textit{A} & \textit{A} &
  \textit{A} &  \textit{I} \\       
  \hline
  10 Minutes &  
  \shortstack{C1-A: \xmark \\ C2-A: \xmark \\ C1-D: \xmark \\ C2-D: \xmark} & 
  \shortstack{C1-A: \cmark \\ C2-A: \cmark \\ C1-D: \xmark \\ C2-D: \xmark} & 
  \shortstack{C1-A: \cmark \\ C2-A: \cmark \\ C1-D: \xmark \\ C2-D: \xmark} & 
  \shortstack{C1-A: \xmark \\ C2-A: \xmark \\ C1-D: \cmark \\ C2-D: \cmark} \\
  \hline          
  \textit{Decision} & \textit{I} & \textit{A} &
  \textit{A} &  \textit{D} \\       
  \hline  
  15 Minutes &  
  \shortstack{C1-A: \xmark \\ C2-A: \xmark \\ C1-D: \xmark \\ C2-D: \xmark} & 
  \shortstack{C1-A: \cmark \\ C2-A: \cmark \\ C1-D: \xmark \\ C2-D: \xmark} & 
  \shortstack{C1-A: \cmark \\ C2-A: \cmark \\ C1-D: \xmark \\ C2-D: \xmark} & 
  \shortstack{C1-A: \xmark \\ C2-A: \xmark \\ C1-D: \cmark \\ C2-D: \cmark} \\
  \hline          
  \textit{Decision} & \textit{I} & \textit{A} &
  \textit{A} &  \textit{D} \\        
  \hline  
  30 Minutes &  
  \shortstack{C1-A: \xmark \\ C2-A: \xmark \\ C1-D: \xmark \\ C2-D: \xmark} & 
  \shortstack{C1-A: \cmark \\ C2-A: \cmark \\ C1-D: \xmark \\ C2-D: \xmark} & 
  \shortstack{C1-A: \xmark \\ C2-A: \cmark \\ C1-D: \xmark \\ C2-D: \xmark} & 
  \shortstack{C1-A: \xmark \\ C2-A: \xmark \\ C1-D: \cmark \\ C2-D: \cmark} \\
  \hline          
  \textit{Decision} & \textit{I} & \textit{A} &
  \textit{I} &  \textit{D} \\    
  \hline     
  \hline  
  \textbf{Final decision} & \textbf{I} & \textbf{A} &
  \textbf{A} &  \textbf{D} \\
  \hline    
  \hline
  \end{tabular}
\end{table}

From Table~\ref{rq3} we observe that file switches, and namespace switches are advatageous for productvity, whereas, state switches are disadvantageous. According to our analysis from non linear correlation coeffcients all switches were marked as inert for all five windows. 










\section{Discussion}
In this section we discuss our findings by stating the implications of our findings for programmers and researchers.  

%\subsection{Programmer Perspective}
%We state the following implications that might be relevant to programmers: 

\begin{itemize}
\begin{item}
From Figure 1 we observe the count of high productive sessions drops when window duration is greater than five or more minutes. The count of high productive sessions drops to as low as 0.65\%
when considering a window duration of 30 minutes. With the increase in window duration count of hih productive sessions decreases. As we use count of programming actions to determine productivity, this finding suggests that 
programmers perform more programming related tasks for smaller periods of time. 

\fbox{\begin{minipage}{22em}
\textbf{Observation-1}: Programmers do not engage heavily in programming related tasks for longer periods of time e.g. for a duration of 15 or 30 minutes. 
\end{minipage}}
     
\end{item}

\begin{item}
Compared to the amount of file and namespace switches the amount of 
domain switches were larger and had a wider spread. According to our definition and assumption detecion of domain switches is less restrictive 
than that of file and namespace switches, and this factor can have an impact on the findings. 

We observe the mean and median amount of state switching is smaller than that of file switches for four of the five windows. Compared to file swithes the detection procedure is less restrictive and this finding suggests that compared to artifact switching such as file switches, state switching happens less frequently when programmners are highly productive.  

\fbox{\begin{minipage}{22em}
\textbf{Observation-2}: Programmers switch less frequently between programming activities such as coding and debugging when they are more productive. 
\end{minipage}}
     
\end{item}

\begin{item}

From our analysis of RQ3 we observe that file and namespace switches 
are idnetified as advantagous switches, whereas state switches are identified as disadvanatageous. From Table 2 we also observe that state switches have been identified for windows that have duration of 10 minutes or more, even though file and namespace switches have been identifed as advantageous for at least four windows. We also observe that the count of programmers that belong in the `negative' category increases from 54 to 72, when window duraion is increased from five minutes to 10 minutes. These findings suggest that the disadvatagous effects of state switching can be realized when programmers 
remain highly productive for longer periods of time.        
       
\fbox{\begin{minipage}{22em}
\textbf{Observation-3}:  
State switches might effect more negatively on productivity when programmers are more productive for larger durations. 
\end{minipage}}
     
\end{item}

\begin{item}
We acknowledge that our detection and quantifying procedure of domain switching is less restrictive than that of file and namespaces switches. From the previous section we have observed how wide the variability is the amount of domain switches compared to file and namespace switches. With respect tovariability, the amount of file and namespace switches might benefit from the comparatively restricted detection and quantifying procedure. We do not claim that our approach is perfect and we observe the necessity of performing future research studies in this direction.                   
       
\fbox{\begin{minipage}{22em}
\textbf{Observation-4}:  
Future research studies can investigate even further to indentify the appropriate procedure for detecting and quantifying different switches.      
\end{minipage}}
     
\end{item}



\begin{item}

We have observed specific trends for file, namespace, and state switches from our analysis of RQ2. These trends hold for a subset of the 339 programmers but not all the 339 programmers. Ideally we would expect the trends to hold for all 339 programmers. We performed hypothesis tests to investigate why the trends are holding for a specific set of programmers only. According to our hypothesis test, we can state with 99\% statistical confidence that the average count of high productive sessions was                        
different between the two groups of programmers: one group of programmers for which we observed trends and the other group for which we did not observe any trend. We identify average count of high productive sessions as one of the factors that can expain the difference. 

But there can be other clues specific to the programmers' nature of productivity that we did not explore in this paper.  
        
\fbox{\begin{minipage}{22em}
\textbf{Observation-5}:  

Future empirical studies can investigate if a specific switch has generalizable impact across all programmers and the factors that explain       
such phenomenon. 
\end{minipage}}
     
\end{item}


\begin{item}  

From Table 2 we also observe that file and namespace swithes are idnetified advanatageous for four or more windows. For state switches we have a different observation. State switches have been identified 
disadvantageous for windows that have a duration of 10 or more minutes. This finding suggests that to assess the effect for some switches require longer periods of time. For example, if we have used windows of duration less than ten minutes then we could ahve identified state switches as adavantageous or inert.    
        
\fbox{\begin{minipage}{22em}
\textbf{Observation-6}:  
Spcific types of switches may exhibit their effects for larger durations of time. Thus, experimentation with various durations might assist in assessing
the effects of different switches effectively and accurately.       
\end{minipage}}
     
\end{item}

\end{itemize}


%From our analysis of general correlation, and correlation analysis of
%individual users we have observed a relatively stronger correlation
%between namespace switching and high productivity, as well as file switching,
%and high productivity. We also have concluded that to observe the correlation
%of different types of switching and high productivity, a relatively larger window size
%can be helpful. From these observations we investigate the frequency of different
%types of context switching that can happen for a specific window size. As an illustrative
%example, we discuss relevant findings for a window of 10 minutes.
%
%For a window of 10 minutes, for 339 users we have observed a correlation co-efficient
%of 0.36 between frequency of namespace switching and high productivity. For the same window
%duration we also have observed that the 80[need exact count] for 80 users the correlation co-efficient
%is greater than or equal to 0.50. From Figure~\ref{disc_p_ns} for these users we observe that the
%average, and median frequency of namespace switching both is five[need exact], in case of high productive sessions. For 60[need exact count] users the correlation coefficient of interest lies between
%0.00 and 0.499, and for these users the average, and median both lies around two for
%high productive sessions. For 10[need exact count] users the correlation coefficient of interest is negative and for these users the average, and median both is less than two. Using Vargha and Delaney's A12 test statistic~\cite{vargha:a12}, the statistical difference between the frequency of namespace switching for negatively correlated users is `large', and `medium' respectively to that of the frequency of namespace switching for 80 users, and 60 users. This finding
%hints that programmers' switch namespaces at different frequencies, and for this dataset we observe the frequency to be around five for a ten minute window.
%
%\begin{figure}[htbp]
%\centering
%\includegraphics[scale=0.30]{img/disc_p_ns.png}
%%\includegraphics[scale=0.30]{img_pdf/disc_p_ns.pdf}
%\caption{Frequency of Namespace Switching for High Productive Sessions}
%\label{disc_p_ns}
%\end{figure}
%
%For a window of 10 minutes, for 339 users we have observed a correlation co-efficient
%of 0.40 between frequency of file switching and high productivity. We have observed that for 120[need exact count] users the correlation co-efficient
%is greater than or equal to 0.50. From Figure~\ref{disc_p_fs} for these users we observe that the
%average, and median frequency of namespace switching respectively  is 20[need exact], in case of high productive sessions. For 80[need exact count] users the correlation coefficient of interest lies between
%0.00 and 0.499, and for these users the average, and median is respectively 20[need exact] for
%high productive sessions. For five[need exact count] users the correlation coefficient of interest is negative and for these users the average, and median both is 20. Using Vargha and Delaney's A12 test statistic~\cite{vargha:a12}, the statistical difference between the frequency of namespace switching for negatively correlated users is `small' for both cases: the frequency of namespace switching for 120 users, and 80 users. This finding hints that programmers' switches from one file to another at a similar frequency, and for this dataset we observe it that to be around 20 for a ten minute window.
%
%\begin{figure}[htbp]
%\centering
%\includegraphics[scale=0.30]{img/disc_p_fs.png}
%%\includegraphics[scale=0.30]{img_pdf/disc_p_fs.pdf}
%\caption{Frequency of File Switching for High Productive Sessions}
%\label{disc_p_ns}
%\end{figure}
%
%For a window of 10 minutes, for 339 users we have observed a correlation co-efficient
%of 0.09 between frequency of domain switching and high productivity. Compared to namespace switching, and file
%switching the correlation co-efficient is small, indicating a weak correlation between
%high productivity and domain switching. We also have observed that the count of users
%for which the correlation co-efficient is greater than 0.50 is only 40 out of 339 users.
%From Figure~\ref{disc_p_ps} for these users we observe that the
%average, and median frequency of namespace switching respectively  is 20[need exact]. For 40[need exact count] users the correlation coefficient of interest lies between
%0.00 and 0.499, and for these users the average, and median is respectively 20[need exact] for
%high productive sessions. For 30[need exact count] users the correlation coefficient of interest is negative and for these users the average, and median both is 20. Using Vargha and Delaney's A12 test statistic~\cite{vargha:a12}, the statistical difference between the frequency of namespace switching for negatively correlated users is `medium' for both cases: the frequency of domain switching for the other two sets of users. This finding hints that programmers' switches from one project context or domain to another at different frequencies.
%
%\begin{figure}[htbp]
%\centering
%\includegraphics[scale=0.30]{img/disc_p_ds.png}
%%\includegraphics[scale=0.30]{img_pdf/disc_p_ds.pdf}
%\caption{Frequency of Domain Switching for High Productive Sessions}
%\label{disc_p_ds}
%\end{figure}
%
%Namespace switching is detected when programmer switches from one namespace to another inside a file. File switching is detected when programmer switches from file another inside a project context or domain. For file switching and namespace switching, we have observed higher general correlations compared to that of domain switching. Also, the number of users showing a correlation greater than 0.50 is higher for both types
%of switching than that of domain switching. This finding indicates that while maintaining high productivity programmers switch between artifacts that belong to the same domain or project context. [Can we add related findings from a relevant paper or other resource? - Dr.Parnin] Future studies can investigate if similar observation holds using different datasets or a different productivity calculation method. If this particular observation holds then programmers can be encouraged to switch more at local levels of project artifacts and switch less in between domains. In a similar fashion empirical studies can also recommend how frequently programmers can switch between files, namespaces or domains.
%
%Compared to namespace switching we also observe the frequencies of file switching and domain switching to be higher for a window of ten minutes. This observation indicates that while maintaining high productivity programmers' perform tasks at small segments of code that can span across different files or different domains. [Can we add related findings from a relevant paper or other resource? - Dr.Parnin] If future or existing empirical studies
%confirm this finding then programmers can be encouraged to work on smaller segments of code residing in a single
%file while remaining highly productive. IDE plugins can track relevant information and recommend programmers to focus their effort on smaller pieces of code rather than writing
%larger chunks of code when they are highly productive.
%
%
%\subsection{What is the Frequency of Switching Between States When Programmers are Highly Productive?}
%
%For a window of 10 minutes, for 339 users we have observed a correlation co-efficient
%of -0.22 between frequency of state switching and high productivity. The correlation co-efficient is negative
%for this type of context switching indicating switching between programming states namely coding, debugging, and building can be counter productive. We also have observed that the count of users
%for which the correlation co-efficient is greater than 0.50 is only five out of 339 users, whereas the
%correlation co-efficient is negative for 99[need exact] users.
%From Figure~\ref{disc_p_ss} we observe that the
%average, and median frequency of namespace switching for these users respectively  is five, and nine[need exact]. For 11[need exact count] users the correlation coefficient of interest lies between
%0.00 and 0.499, and for these users the average, and median is respectively three, and seven[need exact] for
%high productive sessions. For 90[need exact count] users the correlation coefficient of interest is negative and for these users the average, and median respectively is four, and 12 [need exact]. Using Vargha and Delaney's A12 test statistic, the statistical difference between the frequency of namespace switching for negatively correlated users is `small' for both cases: the frequency of state switching for the other two sets of users. Statistically
%we conclude that there is no significant support that the frequency of switching states differs from one group
%of users to other, we observe that the average frequency of state switching occurs is widely
%spread for users that show a negative correlation between high productivity and state switching. Our empirical findings 
%state that 46.38\%, 23.49\%, and 11.67\% of the sessions are respectively categorized as `Idle', `Code', and `Debug'. From this observation 
%we state that when switching states, programmers' tend to remain in any of these three states while remaining highly productive.    
%If future studies can replicate these observations then programmers can be encouraged to involve more in coding and
%debugging activities to remain highly productive.  
%Future studies can investigate these observations, and if these observations hold across significant empirical studies we
%can recommend programmers to do less amount of state switching when they are in highly productive mode.
%IDE plugins can track programmers' switching of states for specific time windows, collect appropriate
%metrics, and alert programmers accordingly.  
%
%
%
%\begin{figure}[htbp]
%\centering
%\includegraphics[scale=0.30]{img/disc_p_ss.png}
%%\includegraphics[scale=0.30]{img_pdf/disc_p_ss.pdf}
%\caption{Frequency of State Switching for High Productive Sessions}
%\label{disc_p_ss}
%\end{figure}

%\subsection{Research Aspects} 

\section{Conclusion}
Conclusion goes here.

%ACKNOWLEDGMENTS are optional
%\section{Acknowledgments}
%This section is optional; it is a location for you
%to acknowledge grants, funding, editing assistance and
%what have you.  In the present case, for example, the
%authors would like to thank Gerald Murray of ACM for
%his help in codifying this \textit{Author's Guide}
%and the \textbf{.cls} and \textbf{.tex} files that it describes.

%
% The following two commands are all you need in the
% initial runs of your .tex file to
% produce the bibliography for the citations in your paper.
\bibliographystyle{abbrv}
\bibliography{icpc2016}
% You must have a proper ".bib" file
%  and remember to run:
% latex bibtex latex latex
% to resolve all references
%
% ACM needs 'a single self-contained file'!
%
%APPENDICES are optional
%\balancecolumns

\balancecolumns
% That's all folks!
\end{document}
